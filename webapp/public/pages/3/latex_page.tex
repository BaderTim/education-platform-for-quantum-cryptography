\section{Quantum Computing}

Die fortschreitende Digitalisierung bringt laufend neue Erfindungen und Gerätschaften mit sich. Zuerst war es ein Taschenrechner, dann der Bürocomputer und nun sogar schon smarte Kühlschränke. Doch egal wie unterschiedlich diese Geräte auch sein mögen, sie alle haben eine grundlegende Gemeinsamkeit: 1 und 0 - oder auch AN und AUS. Diese Information wird in klassischen Computersystemen auch als Bit bezeichnet, welches sich durch elektrische Ladung entweder in dem Zustand 1 oder 0 befindet. Wenige Bits können zusammen mit logischen Gattern bereits einfache Berechnungen durchführen, was bei einer höheren Skalierung und komplexeren Schaltkreisen letztendlich zu den heutigen Informationsverarbeitungssystemen führt. \newline

Quantum Computing baut auf einer ähnlichen Grundlage wie die klassischen Computersysteme auf, doch verwendet anstelle von herkömmlichen Bits \textit{Quantum Bits}, die auf den Prinzipien der Quantenmechanik basieren. Durch Phänomene wie die der Superposition und des \textit{Quantum Entanglements} können Schaltkreise nach einer anderen Logik entwickelt werden, die bei korrekter Anwendung oftmals enorme Vorteile gegenüber den klassischen Systemen vorweisen. (Nielsen und Chuang, 2001, Vgl. S. 4-5) \newline
In den folgenden Untersektionen wird die Funktionsweise von Quantensystemen, und wie sie dadurch Vorteile erzielen, erläutert.

\newline \newline
\exercise[type=multipleChoice]{
    \question{Frage: Bieten Quantencomputer immer Vorteile gegenüber klassischen Computersystemen? }
    \possibleAnswers{
        \item 1) Quantencomputer sind prinzipiell schneller als klassische Computersysteme.
        \item 2) Nein, Quantencomputer können gar nicht schneller als die heutigen Rechensysteme sein.
        \item 3) Das ist vom jeweiligem Anwendungsgebiet abhängig.
    }
    \result{3}
}



\newline \newline
\subsection{Quellen}
[Nielsen und Chuang 2001] Nielsen, Michael A. ; Chuang, Isaac L.: Quantum computation and quantum information. In: Phys. Today 54 (2001), Nr. 2, S. 60\newline